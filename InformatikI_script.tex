\documentclass[paper=a4, fontsize=11pt]{scrartcl} 
\usepackage[utf8]{inputenc}
\usepackage{amsmath}
\usepackage{amsfonts}
\usepackage{amssymb}
\author{Kim Thuong Ngo}
\usepackage[T1]{fontenc} 
\usepackage{fourier} 
\usepackage{lipsum} 
\usepackage{listings}
\usepackage{graphicx}
\usepackage{tabularx}
\usepackage{sectsty}
\usepackage{xcolor}
\allsectionsfont{\centering \normalfont\scshape} 
\usepackage{fancyhdr} 
\pagestyle{fancyplain} 
\fancyhead{}
\fancyfoot[L]{} 
\fancyfoot[C]{} 
\fancyfoot[R]{\thepage} 
\renewcommand{\headrulewidth}{0pt} 
\renewcommand{\footrulewidth}{0pt}
\setlength{\headheight}{13.6pt}

\numberwithin{equation}{section} 
\numberwithin{figure}{section} 
\numberwithin{table}{section}

\setlength\parindent{0pt}  

\newcommand{\horrule}[1]{\rule{\linewidth}{#1}} 

\title{	
\normalfont \normalsize 
\textsc{InformatikI} \\ [25pt] 
\horrule{0.5pt} \\[0.4cm] 
\huge Skript \\ 
\horrule{2pt} \\[0.5cm] 
}

\author{Kim Thuong Ngo} 
\date{\normalsize\today} 

\usepackage{color}

\definecolor{mygreen}{rgb}{0,0.6,0}
\definecolor{mygray}{rgb}{0.83,0.83,0.83}

\lstset{
   language={Scheme},
   basicstyle=\small,
   keywordstyle=\color{blue},
   identifierstyle=,
   commentstyle=\color{mygreen},
   stringstyle=\ttfamily,
   breaklines=true,
   numbers=left,
   numberstyle=\small,
   frame=single,
   backgroundcolor=\color{mygray}
}

%----------------------------------------------------------------------------------------
\begin{document}
\maketitle 
\newpage
\tableofcontents
%----------------------------------------------------------------------------------------
\newpage
\section{Scheme: Ausdrücke, Auswertung und Abstraktion}
Programm: DrRacket \\

Bild \\

Die Anwendung von Funktionen wird in Scheme \textcolor{orange}{ausschließlich} in \textcolor{orange}{Präfixnotation} durchgeführt:
\begin{tabular}{c|c}
Mathematik & Scheme \\\hline
$44-2$ & (- 44 2) \\
$f(x,y)$ & (f x y)\\
$\sqrt{81}$ & (sqrt 81)\\
$\lfloor x \rfloor$ & (floor x)\\
$9^{2}$  & (expt 9 2) \\
3! & (! 3) \\
\end{tabular}

Allgemein:
\begin{lstlisting}
(<function> <argument> ... <argument> )
\end{lstlisting}

(+ 40 2) und (odd? 42) sind Beispiele für \textcolor{orange}{Ausdrücke}, die bei der \textcolor{orange}{Auswertung} einen Wert liefern.

Notation: $\rightsquigarrow$
\begin{tabular}{cc}
(+ 40 2) $\rightsquigarrow$ 42 & Auswertung/ Evaluation \\
(odd? 42) $\rightsquigarrow$ Reduktion \\
\end{tabular}

Interaktionsfenster : Read $\rightarrow$ Evaluation $\rightarrow$ Print $\rightarrow$ Read (REPL) \\

\textcolor{orange}{Literale}


%----------------------------------------------------------------------------------------

\end{document}